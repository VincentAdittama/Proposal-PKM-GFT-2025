\section*{LAMPIRAN}

\subsection*{Lampiran 1. Biodata Ketua dan Anggota, serta Dosen Pendamping}

{\setlength{\parindent}{0pt}
    \textbf{Biodata Ketua}\par
    \textbf{A. Identitas Diri}\par
    \vspace{6pt}
    \begin{tabularx}{\textwidth}{|c|l|
        >{\raggedright\arraybackslash\hspace{0pt}}X|}
    \hline
    1 & Nama Lengkap             & \ketuaNama   \\ \hline
    2 & Jenis Kelamin            & \ketuaGender \\ \hline
    3 & Program Studi            & \ketuaProdi  \\ \hline
    4 & NIM                      & \ketuaNIM    \\ \hline
    5 & Tempat dan Tanggal Lahir & \ketuaTTL    \\ \hline
    6 & Alamat E-mail            & \ketuaEmail  \\ \hline
    7 & Nomor Telepon/HP         & \ketuaTelp   \\ \hline
    \end{tabularx}\par

    \vspace{0.5cm}
    \textbf{B. Kegiatan Kemahasiswaan yang Sedang/Pernah Diikuti}\par
    \vspace{6pt}
    \begin{tabularx}{\textwidth}{|c|X|X|X|X|}
    \hline
    \multicolumn{1}{|c|}{\textbf{No}} & 
    \multicolumn{1}{>{\centering\arraybackslash}X|}{\makecell{\textbf{Jenis Kegiatan}}} & 
    \multicolumn{1}{>{\centering\arraybackslash}X|}{\makecell{\textbf{Status dalam}\\\textbf{Kegiatan}}} & 
    \multicolumn{1}{>{\centering\arraybackslash}X|}{\makecell{\textbf{Waktu dan Tempat}}} \\ \hline
    1 & Sarjana  & & \\ \hline
    2 & & & \\ \hline
    \end{tabularx}\par

    \vspace{0.5cm}
    \textbf{C. Penghargaan yang Pernah Diterima}\par
    \vspace{6pt}
    \begin{tabularx}{\textwidth}{|c|X|X|X|}
    \hline
    \multicolumn{1}{|c|}{\textbf{No}} & 
    \multicolumn{1}{>{\centering\arraybackslash}X|}{\makecell{\textbf{Jenis}\\\textbf{Penghargaan}}} & 
    \multicolumn{1}{>{\centering\arraybackslash}X|}{\makecell{\textbf{Pihak Pemberi}\\\textbf{Penghargaan}}} & 
    \multicolumn{1}{>{\centering\arraybackslash}X|}{\makecell{\textbf{Tahun}}} \\ \hline
    1 & & & \\ \hline
    2 & & & \\ \hline
    \end{tabularx}\par

    % Statement text at the end of each biodata
    \vspace{0.5cm}
    Semua data yang saya isikan dan tercantum dalam biodata ini adalah benar dan dapat dipertanggungjawabkan secara hukum. Apabila di kemudian hari ternyata dijumpai ketidaksesuaian dengan kenyataan, saya sanggup menerima sanksi.

    \vspace{0.5cm}
    Demikian biodata ini saya buat dengan sebenarnya untuk memenuhi salah satu persyaratan dalam pengajuan \textbf{PKM-GFT}.

    \vspace{1cm}
    \begin{flushright}
    Yogyakarta, \today\\
    Ketua Tim\\
    \vspace{2cm}
    (\ketuaNama)
    \end{flushright}
}
% Ketua to Anggota 1 ------------------------------

\newpage
{\setlength{\parindent}{0pt}
    \textbf{Biodata Anggota 1}\par
    \textbf{A. Identitas Diri}\par
    \vspace{6pt}
    \begin{tabularx}{\textwidth}{|c|l|
        >{\raggedright\arraybackslash\hspace{0pt}}X|}
    \hline
    1 & Nama Lengkap             & \anggotaSatuNama   \\ \hline
    2 & Jenis Kelamin            & \anggotaSatuGender \\ \hline
    3 & Program Studi            & \anggotaSatuProdi  \\ \hline
    4 & NIM                      & \anggotaSatuNIM    \\ \hline
    5 & Tempat dan Tanggal Lahir & \anggotaSatuTTL    \\ \hline
    6 & Alamat E-mail            & \anggotaSatuEmail  \\ \hline
    7 & Nomor Telepon/HP         & \anggotaSatuTelp   \\ \hline
    \end{tabularx}\par

    \vspace{0.5cm}
    \textbf{B. Kegiatan Kemahasiswaan yang Sedang/Pernah Diikuti}\par
    \vspace{6pt}
    \begin{tabularx}{\textwidth}{|c|X|X|X|}
    \hline
    \multicolumn{1}{|c|}{\textbf{No}} & 
    \multicolumn{1}{>{\centering\arraybackslash}X|}{\makecell{\textbf{Jenis Kegiatan}}} & 
    \multicolumn{1}{>{\centering\arraybackslash}X|}{\makecell{\textbf{Status dalam}\\\textbf{Kegiatan}}} & 
    \multicolumn{1}{>{\centering\arraybackslash}X|}{\makecell{\textbf{Waktu dan Tempat}}} \\ \hline
    1 & & & \\ \hline
    2 & & & \\ \hline
    \end{tabularx}\par

    \vspace{0.5cm}
    \textbf{C. Penghargaan yang Pernah Diterima}\par
    \vspace{6pt}
    \begin{tabularx}{\textwidth}{|c|X|X|X|}
    \hline
    \multicolumn{1}{|c|}{\textbf{No}} & 
    \multicolumn{1}{>{\centering\arraybackslash}X|}{\makecell{\textbf{Jenis}\\\textbf{Penghargaan}}} & 
    \multicolumn{1}{>{\centering\arraybackslash}X|}{\makecell{\textbf{Pihak Pemberi}\\\textbf{Penghargaan}}} & 
    \multicolumn{1}{>{\centering\arraybackslash}X|}{\makecell{\textbf{Tahun}}} \\ \hline
    1 & & & \\ \hline
    2 & & & \\ \hline
    \end{tabularx}\par

    % Statement text at the end of each biodata
    \vspace{0.5cm}
    Semua data yang saya isikan dan tercantum dalam biodata ini adalah benar dan dapat dipertanggungjawabkan secara hukum. Apabila di kemudian hari ternyata dijumpai ketidaksesuaian dengan kenyataan, saya sanggup menerima sanksi.

    \vspace{0.5cm}
    Demikian biodata ini saya buat dengan sebenarnya untuk memenuhi salah satu persyaratan dalam pengajuan \textbf{PKM-GFT}.

    \vspace{1cm}
    \begin{flushright}
    Yogyakarta, \today\\
    Anggota Tim\\
    \vspace{2cm}
    (\anggotaSatuNama)
    \end{flushright}
}

% Anggota 1 to Anggota 2 ------------------------------

\newpage
{\setlength{\parindent}{0pt}
    \textbf{Biodata Anggota 2}\par
    \textbf{A. Identitas Diri}\par
    \vspace{6pt}
    \begin{tabularx}{\textwidth}{|c|l|
        >{\raggedright\arraybackslash\hspace{0pt}}X|}
    \hline
    1 & Nama Lengkap             & \anggotaDuaNama   \\ \hline
    2 & Jenis Kelamin            & \anggotaDuaGender \\ \hline
    3 & Program Studi            & \anggotaDuaProdi  \\ \hline
    4 & NIM                      & \anggotaDuaNIM    \\ \hline
    5 & Tempat dan Tanggal Lahir & \anggotaDuaTTL    \\ \hline
    6 & Alamat E-mail            & \anggotaDuaEmail  \\ \hline
    7 & Nomor Telepon/HP         & \anggotaDuaTelp   \\ \hline
    \end{tabularx}\par

    \vspace{0.5cm}
    \textbf{B. Kegiatan Kemahasiswaan yang Sedang/Pernah Diikuti}\par
    \vspace{6pt}
    \begin{tabularx}{\textwidth}{|c|X|X|X|}
    \hline
    \multicolumn{1}{|c|}{\textbf{No}} & 
    \multicolumn{1}{>{\centering\arraybackslash}X|}{\makecell{\textbf{Jenis Kegiatan}}} & 
    \multicolumn{1}{>{\centering\arraybackslash}X|}{\makecell{\textbf{Status dalam}\\\textbf{Kegiatan}}} & 
    \multicolumn{1}{>{\centering\arraybackslash}X|}{\makecell{\textbf{Waktu dan Tempat}}} \\ \hline
    1 & & & \\ \hline
    2 & & & \\ \hline
    \end{tabularx}\par

    \vspace{0.5cm}
    \textbf{C. Penghargaan yang Pernah Diterima}\par
    \vspace{6pt}
    \begin{tabularx}{\textwidth}{|c|X|X|X|}
    \hline
    \multicolumn{1}{|c|}{\textbf{No}} & 
    \multicolumn{1}{>{\centering\arraybackslash}X|}{\makecell{\textbf{Jenis}\\\textbf{Penghargaan}}} & 
    \multicolumn{1}{>{\centering\arraybackslash}X|}{\makecell{\textbf{Pihak Pemberi}\\\textbf{Penghargaan}}} & 
    \multicolumn{1}{>{\centering\arraybackslash}X|}{\makecell{\textbf{Tahun}}} \\ \hline
    1 & & & \\ \hline
    2 & & & \\ \hline
    \end{tabularx}\par

    % Statement text at the end of each biodata
    \vspace{0.5cm}
    Semua data yang saya isikan dan tercantum dalam biodata ini adalah benar dan dapat dipertanggungjawabkan secara hukum. Apabila di kemudian hari ternyata dijumpai ketidaksesuaian dengan kenyataan, saya sanggup menerima sanksi.

    \vspace{0.5cm}
    Demikian biodata ini saya buat dengan sebenarnya untuk memenuhi salah satu persyaratan dalam pengajuan \textbf{PKM-GFT}.

    \vspace{1cm}
    \begin{flushright}
    Yogyakarta, \today\\
    Anggota Tim\\
    \vspace{2cm}
    (\anggotaDuaNama)
    \end{flushright}
}

% Anggota 2 to Anggota 3 ------------------------------

\newpage
{\setlength{\parindent}{0pt}
    \textbf{Biodata Anggota 2}\par
    \textbf{A. Identitas Diri}\par
    \vspace{6pt}
    \begin{tabularx}{\textwidth}{|c|l|
        >{\raggedright\arraybackslash\hspace{0pt}}X|}
    \hline
    1 & Nama Lengkap             & \anggotaTigaNama   \\ \hline
    2 & Jenis Kelamin            & \anggotaTigaGender \\ \hline
    3 & Program Studi            & \anggotaTigaProdi  \\ \hline
    4 & NIM                      & \anggotaTigaNIM    \\ \hline
    5 & Tempat dan Tanggal Lahir & \anggotaTigaTTL    \\ \hline
    6 & Alamat E-mail            & \anggotaTigaEmail  \\ \hline
    7 & Nomor Telepon/HP         & \anggotaTigaTelp   \\ \hline
    \end{tabularx}\par

    \vspace{0.5cm}
    \textbf{B. Kegiatan Kemahasiswaan yang Sedang/Pernah Diikuti}\par
    \vspace{6pt}
    \begin{tabularx}{\textwidth}{|c|X|X|X|}
    \hline
    \multicolumn{1}{|c|}{\textbf{No}} & 
    \multicolumn{1}{>{\centering\arraybackslash}X|}{\makecell{\textbf{Jenis Kegiatan}}} & 
    \multicolumn{1}{>{\centering\arraybackslash}X|}{\makecell{\textbf{Status dalam}\\\textbf{Kegiatan}}} & 
    \multicolumn{1}{>{\centering\arraybackslash}X|}{\makecell{\textbf{Waktu dan Tempat}}} \\ \hline
    1 & & & \\ \hline
    2 & & & \\ \hline
    \end{tabularx}\par

    \vspace{0.5cm}
    \textbf{C. Penghargaan yang Pernah Diterima}\par
    \vspace{6pt}
    \begin{tabularx}{\textwidth}{|c|X|X|X|}
    \hline
    \multicolumn{1}{|c|}{\textbf{No}} & 
    \multicolumn{1}{>{\centering\arraybackslash}X|}{\makecell{\textbf{Jenis}\\\textbf{Penghargaan}}} & 
    \multicolumn{1}{>{\centering\arraybackslash}X|}{\makecell{\textbf{Pihak Pemberi}\\\textbf{Penghargaan}}} & 
    \multicolumn{1}{>{\centering\arraybackslash}X|}{\makecell{\textbf{Tahun}}} \\ \hline
    1 & & & \\ \hline
    2 & & & \\ \hline
    \end{tabularx}\par

    % Statement text at the end of each biodata
    \vspace{0.5cm}
    Semua data yang saya isikan dan tercantum dalam biodata ini adalah benar dan dapat dipertanggungjawabkan secara hukum. Apabila di kemudian hari ternyata dijumpai ketidaksesuaian dengan kenyataan, saya sanggup menerima sanksi.

    \vspace{0.5cm}
    Demikian biodata ini saya buat dengan sebenarnya untuk memenuhi salah satu persyaratan dalam pengajuan \textbf{PKM-GFT}.

    \vspace{1cm}
    \begin{flushright}
    Yogyakarta, \today\\
    Anggota Tim\\
    \vspace{2cm}
    (\anggotaTigaNama)
    \end{flushright}
}

% Anggota 3 to Dosen ------------------------------

\newpage
{\setlength{\parindent}{0pt}
    \textbf{Biodata Dosen Pendamping}\par
    \textbf{A. Identitas Diri}\par
    \vspace{6pt}
    \begin{tabularx}{\textwidth}{|c|l|
        >{\raggedright\arraybackslash\hspace{0pt}}X|}
    \hline
    1 & Nama Lengkap (dengan gelar) & \dosenNama   \\ \hline
    2 & Jenis Kelamin               & \dosenGender \\ \hline
    3 & Program Studi               & \dosenProdi  \\ \hline
    4 & NIM                         & \dosenNIDN    \\ \hline
    5 & Tempat dan Tanggal Lahir    & \dosenTTL    \\ \hline
    6 & Alamat E-mail               & \dosenEmail  \\ \hline
    7 & Nomor Telepon/HP            & \dosenTelp   \\ \hline
    \end{tabularx}\par

    \vspace{0.5cm}
    \textbf{B. Riwayat Pendidikan}\par
    \vspace{6pt}
    \begin{tabularx}{\textwidth}{|c|X|X|c|c|}
    \hline
    \multicolumn{1}{|c|}{\textbf{No}} & 
    \multicolumn{1}{>{\centering\arraybackslash}X|}{\makecell{\textbf{Jenjang}}} & 
    \multicolumn{1}{>{\centering\arraybackslash}X|}{\makecell{\textbf{Bidang Ilmu}}} & 
    \multicolumn{1}{>{\centering\arraybackslash}X|}{\makecell{\textbf{Institusi}}} & 
    \multicolumn{1}{>{\centering\arraybackslash}X|}{\makecell{\textbf{Tahun Lulus}}} \\ \hline
    1 & Sarjana (S1) & Musik & ISI Yogyakarta & 1985\\ \hline
    2 & Magister (S2) & Humaniora/ Pengkajian Seni Pertunjukan & UGM & 1994 \\ \hline
    2 & Doktor (S3) & Humanioara/ Pengkajian Seni Pertunjukan dan Seni Rupa & UGM & 2010\\ \hline
    \end{tabularx}\par

    \vspace{0.5cm}
    \textbf{C. Rekam Jejak Tri Dharma PT (dalam 5 tahun terakhir)}\par
    \vspace{6pt}
    Pendidikan/Pengajaran\par
    \begin{tabularx}{\textwidth}{|c|X|l|l|}
        \hline
        \multicolumn{1}{|c|}{\textbf{No}} & 
        \multicolumn{1}{>{\centering\arraybackslash}X|}{\makecell{\textbf{Nama Mata Kuliah}}} & 
        \multicolumn{1}{>{\centering\arraybackslash}l|}{\makecell{\textbf{Wajib/Pilihan}}} & 
        \multicolumn{1}{>{\centering\arraybackslash}l|}{\makecell{\textbf{SKS}}} \\ \hline
        1 & Analisis Bentuk Musik & Wajib & 3\\ \hline
        2 & Kajian Literatur Musik & Wajib & 3\\ \hline
        3 & Musikologi III & Wajib & 2\\ \hline
    \end{tabularx}\par
    \vspace{0.5cm}
    
    Penelitian\par
    \begin{tabularx}{\textwidth}{|c|X|X|l|}
        \hline
        \multicolumn{1}{|c|}{\textbf{No}} & 
        \multicolumn{1}{>{\centering\arraybackslash}X|}{\makecell{\textbf{Judul Penelitian}}} & 
        \multicolumn{1}{>{\centering\arraybackslash}l|}{\makecell{\textbf{Penyandang Dana}}} & 
        \multicolumn{1}{>{\centering\arraybackslash}l|}{\makecell{\textbf{Tahun}}} \\ \hline
        1 & Kajian Historis Pembelajaran Gitar Klasik di Jurusan Musik FSP ISI Yogyakarta & ISI Yogyakarta& 2021\\ \hline
        2 & Kajian Musikologis Lantunan Surat al Fatihah oleh Imam Tetap Masjid Jogokariyan Yogyakarta & ISI Yogyakarta & 2022\\ \hline
        3 & Inovasi Asesmen Pembelajaran Kelas untuk Peningkatan Akurasi Penilaian Akhir Kuliah Analisis Bentuk Musik di Perguruan Tinggi Seni & ISI Yogyakarta & 2024\\ \hline
        4 & Hubungan antara Identitas Sosial dan Atribusi dengan Preferensi Musik pada Kalangan Remaja di Yogyakarta (anggota peneliti) & Kemdikbudristek/DRPM & 20204\\ \hline
    \end{tabularx}\par
    \vspace{0.5cm}

    Pengabdian Kepada Masyarakat\par
    \begin{tabularx}{\textwidth}{|c|X|X|l|}
        \hline
        \multicolumn{1}{|c|}{\textbf{No}} & 
        \multicolumn{1}{>{\centering\arraybackslash}X|}{\makecell{\textbf{Judul Pengabdian}\\\textbf{Kepada Masyarakat}}} & 
        \multicolumn{1}{>{\centering\arraybackslash}l|}{\makecell{\textbf{Penyandang Dana}}} & 
        \multicolumn{1}{>{\centering\arraybackslash}l|}{\makecell{\textbf{Tahun}}} \\ \hline
        1 & Editor Jurnal Promusika & ISI Yogyakarta& 2024\\ \hline
    \end{tabularx}\par
    \vspace{0.5cm}

    % Statement text at the end of each biodata
    \vspace{0.5cm}
    Semua data yang saya isikan dan tercantum dalam biodata ini adalah benar dan dapat dipertanggungjawabkan secara hukum. Apabila di kemudian hari ternyata dijumpai ketidaksesuaian dengan kenyataan, saya sanggup menerima sanksi.

    \vspace{0.5cm}
    Demikian biodata ini saya buat dengan sebenarnya untuk memenuhi salah satu persyaratan dalam pengajuan \textbf{PKM-GFT}.

    \vspace{1cm}
    \begin{flushright}
    Yogyakarta, \today\\
    Dosen Pendamping\\
    \vspace{2cm}
    (\dosenNama)
    \end{flushright}
}

\newpage
\subsection*{Lampiran 2. Susunan Tim Pengusul dan Pembagian Tugas}
\begin{flushleft}
    % Add team structure and task distribution content here
\end{flushleft}

\newpage
\subsection*{Lampiran 3. Surat Pernyataan Ketua Tim Pengusul}
\begin{flushleft}
    % Add declaration letter content here
\end{flushleft}