\documentclass[12pt,a4paper]{article}

% Required packages
\usepackage[left=4cm,right=3cm,top=3cm,bottom=3cm]{geometry} % Margins
\usepackage{times}               % Times New Roman font
\usepackage[utf8]{inputenc}      % Input encoding for UTF-8 characters
\usepackage[T1]{fontenc}         % Font encoding for proper character rendering
\usepackage[indonesian]{babel}   % Indonesian language support
\usepackage{setspace}            % For line spacing
\usepackage{ragged2e}            % For text justification
\usepackage{tocloft}             % For table of contents formatting
\usepackage{titlesec}            % For section formatting
\usepackage{lipsum}              % For lorem ipsum
\usepackage[hidelinks]{hyperref} % For clickable links in TOC
\usepackage{graphicx}            % Package for handling graphics/images
\usepackage{scrlayer-scrpage}    % Package for header and footer customization
\usepackage[hidelinks]{hyperref} % Package for clickable links without visible borders
\usepackage{tabularx}            % Package for advanced table formatting with auto-adjusting columns
\usepackage{makecell}            % Package for creating cells with line breaks in tables
\usepackage[none]{hyphenat}      % Package to disable hyphenation
\usepackage{showframe}

% Document settings
\setstretch{1.15}            % 1.15 line spacing
% \justifying                  % Justified text alignment
\newcommand{\judul}{<Judul Kita di Sini>}

% Dosen Pendamping Variables
\newcommand{\dosenNama}{Prof. Dr. Andre Indrawan, M.Hum., M.Mus.}
\newcommand{\dosenSingkatNama}{Andre Indrawan}
\newcommand{\dosenGender}{Laki-laki}
\newcommand{\dosenProdi}{Musik (S1)}
\newcommand{\dosenNIP}{196105101987031002}
\newcommand{\dosenNIDN}{0010056110}
\newcommand{\dosenTTL}{Bandung, 10 Mei 1961}
\newcommand{\dosenEmail}{indrawan\_andre@isi.ac.id}
\newcommand{\dosenTelp}{+62 818 0425 1709}

% Ketua Kelompok Variables
\newcommand{\ketuaNama}{Vincent Nuridzati Adittama}
\newcommand{\ketuaGender}{Laki-laki}
\newcommand{\ketuaProdi}{Musik (S1)}
\newcommand{\ketuaNIM}{23104810131}
\newcommand{\ketuaTTL}{Trenggalek, 23 November 2004}
\newcommand{\ketuaEmail}{eezaateedanvers@gmail.com}
\newcommand{\ketuaTelp}{+62 822 4697 9795}

% Anggota 1 Variables
\newcommand{\anggotaSatuNama}{Azzahra Qurrata A'yun Adindya Irbah Ramadhani}
\newcommand{\anggotaSatuGender}{Perempuan}
\newcommand{\anggotaSatuProdi}{Musik (S1)}
\newcommand{\anggotaSatuNIM}{24106110131}
\newcommand{\anggotaSatuTTL}{Wonosobo, 7 Oktober 2004}
\newcommand{\anggotaSatuEmail}{irbahramadhani7@gmail.com}
\newcommand{\anggotaSatuTelp}{+62 857 4793 0265}

% Anggota 2 Variables
\newcommand{\anggotaDuaNama}{Elizabeth Ardhayu Maheswari}
\newcommand{\anggotaDuaGender}{Perempuan}
\newcommand{\anggotaDuaProdi}{Musik}
\newcommand{\anggotaDuaNIM}{23104330131}
\newcommand{\anggotaDuaTTL}{Lampung, 26 September 2005}
\newcommand{\anggotaDuaEmail}{ardhayu.maheswari@gmail.com}
\newcommand{\anggotaDuaTelp}{+62 813 2964 8109}

% Anggota 3 Variables
\newcommand{\anggotaTigaNama}{Cherish Coo Shalom Putri Betzia}
\newcommand{\anggotaTigaGender}{Perempuan}
\newcommand{\anggotaTigaProdi}{Musik (S1)}
\newcommand{\anggotaTigaNIM}{24106080131}
\newcommand{\anggotaTigaTTL}{Yogyakarta, 17 Agustus 2006}
\newcommand{\anggotaTigaEmail}{cherishalom07@gmail.com}
\newcommand{\anggotaTigaTelp}{+62 878 1808 7178}

% Section formatting
\addto\captionsindonesian{
  \renewcommand{\contentsname}{DAFTAR ISI}
  \renewcommand{\refname}{DAFTAR PUSTAKA}
  \renewcommand{\abstractname}{ABSTRAK}
  \renewcommand{\figurename}{Gambar}
  \renewcommand{\tablename}{Tabel}
}

% Section formatting
\titleformat{\section}{\large\normalfont\bfseries}{BAB \thesection.}{1em}{}
\titleformat{\subsection}{\normalsize\normalfont\bfseries}{\thesubsection.}{1em}{}

% Page numbering and header/footer settings
\clearpairofpagestyles
\cfoot[]{}

% For front matter (Table of Contents)
\newpairofpagestyles{frontmatter}{
    \KOMAoptions{footsepline=false}
    \cfoot[]{}
    \ofoot*{\pagemark}
}

% For main matter (Chapter 1 onwards)
\newpairofpagestyles{mainmatter}{
    \KOMAoptions{headsepline=false}
    \chead[]{}
    \ohead*{\pagemark}
}

% Remove page number from first page
\begin{document}
\thispagestyle{empty}

% Cover page
\begin{center}
    \makebox[\textwidth]{\includegraphics[width=0.4\textwidth]{img/Logo Isi.png}}\par
    \vspace{1cm}
    {\large\bfseries PROPOSAL PROGRAM KREATIVITAS MAHASISWA}\par
    \vspace{1cm}
    {\large\bfseries \judul}\par
    \vspace{1cm}

    {\large\bfseries BIDANG KEGIATAN\\PKM GAGASAN FUTURISTIK TERTULIS}
    \vspace{2cm}

    Disusun oleh:
    \vspace{0.5cm}

    \begin{tabular}{lcl}
        \ketuaNama & \ketuaNIM & Angkatan 2023\\
        \anggotaSatuNama & \anggotaSatuNIM & Angkatan 2024\\
        \anggotaDuaNama & \anggotaDuaNIM & Angkatan 2023\\
        \anggotaTigaNama & \anggotaTigaNIM & Angkatan 2024
    \end{tabular}
    \vspace{2cm}

    {\large\bfseries INSTITUT SENI INDONESIA\\YOGYAKARTA\\\the\year}
\end{center}

\newpage

% Start Roman numerals for front matter
\pagenumbering{roman}
\pagestyle{frontmatter}

% Table of contents
\renewcommand{\cftsecfont}{\normalfont}      % Normal font for section titles
\renewcommand{\cftsecpagefont}{\normalfont}  % Normal font for page numbers
\renewcommand{\cftsecleader}{\cftdotfill{\cftdotsep}} % Dots in TOC

\tableofcontents
\newpage

% Switch to Arabic numerals for main content
\pagenumbering{arabic}
\pagestyle{mainmatter}

% Main content
\section{PENDAHULUAN}

% Subsections can be added here based on specific requirements
\subsection{Latar Belakang}
% Add your background content here

\subsection{Rumusan Masalah}
% Add problem statements here

\subsection{Tujuan}
% Add objectives here

\subsection{Manfaat}
% Add benefits here

\newpage
\section{GAGASAN}
% Subsections can be added here based on specific requirements
\subsection{Kondisi Kekinian Pencetus Gagasan}
% \lipsum[1-3]
% Add current situation analysis here

\subsection{Solusi yang Pernah Ditawarkan}
% Add previous solutions here

\subsection{Gagasan Baru yang Ditawarkan}
% Add your new proposed solution here

\subsection{Pihak yang Dapat Mengimplementasikan Gagasan}
% Add stakeholders and implementation details here

\subsection{Langkah-langkah Strategis Implementasi Gagasan}
% Add strategic implementation steps here

\newpage
\section{KESIMPULAN}

% Add your conclusion content here
\subsection{Gagasan yang Diajukan}
% Summarize the main ideas proposed

\subsection{Teknik Implementasi}
% Summarize the implementation techniques

\subsection{Prediksi Hasil}
% Describe expected outcomes and impact

\newpage

% References and appendix
\addcontentsline{toc}{section}{DAFTAR PUSTAKA}
\section{DAFTAR PUSTAKA}
% Use the following format for references:
% Nama penulis. Tahun. Judul. Tempat terbit: Penerbit.

% Example:
% Doe, John. 2023. Title of the Book. New York: Publisher Name.

\begin{flushleft}
% Add your references here in alphabetical order

\end{flushleft}

\newpage
\addcontentsline{toc}{section}{LAMPIRAN}
\section*{LAMPIRAN}
% Add your appendices here
% You can include additional materials such as:
% - Supporting data
% - Figures and tables
% - Additional calculations
% - Other relevant documents

\begin{flushleft}
% Add your appendix content here

\end{flushleft}

\end{document}
